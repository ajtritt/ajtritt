\documentclass[11pt,a4paper]{scrartcl}

\usepackage[colorlinks = true, urlcolor = blue]{hyperref}
\usepackage{tikz}
\usepackage[margin=0.9in]{geometry}
\newcommand{\termicon}{\includegraphics[scale=0.25]{terminal}\thinspace}
\setkomafont{disposition}{\normalfont\bfseries}
\renewcommand{\thesection}{}% Remove section references...

\title{\emph{De Novo} Genome Assembly using Short Reads}
\author{}
\date{}

\begin{document}
  \maketitle

\section*{Platform}
Exercises for this laboratory will be carried out on a virtual machine running
Ubuntu. Ubuntu is an open source distribution of the Linux operating system.
The majority of bioinformatics tools are written on and developed to run on
Unix-like platforms, such as Linux or Mac OS X, and are not available for
Windows platforms. 

To run the virtual machine, you will use the program VirtualBox, which can be
downloaded at the following website: \url{https://www.virtualbox.org/wiki/Downloads}

The virtual machine image has been prebuilt with the necessary software. The
virtual machine image can be downloaded from the following link:
\url{https://drive.google.com/file/d/0BxoXt7fkRSWvVGo3eUdQRXNyb0U}

Once you have downloaded VirtualBox and the virtual machine image, you will
need to set up the virtual machine within VirtualBox. To do so, click on ``New''
at the top of the VirtualBox Manager, and follow the walk through instructions
to set up the virtual machine. When you get to the VM Name and OS Type step,
set the Operating System to be ``Linux'' and the Version to be ``Ubuntu'' (do not
select ``Ubuntu 64 bit''). Name the image “genome-assembly”. The next step will be
for Memory. I suggest setting this to 1024 MB. Next you select the Virtual Hard
Disk. Make sure the the box for “Start-up Disk” is checked, and select the
option “Use existing hard disk”. The dropdown box should be empty. To add a
disk to this, click on the folder icon (with a green arrow) to the left of the
dropdown box, and locate the virtual machine image you just downloaded.

Now that you have your virtual machine setup, you can boot it up. From the
VirtualBox Manager, double-click on the machine ``genome-assembly'' to boot the
machine.  Once it has completed booting, you will need to login. The username
is \texttt{ubuntu} and the password is \texttt{reverse}. 

When you are signed in, you will need to open Terminal, the command-line
interface. This is where all your work will be carried out. You can open
Terminal by clicking on the icon \termicon               on the left panel. It is highly recommended
that you maximize the screen once you have it opened. This will make it much
easier to read the output of programs you will run. 

Although it is not necessary, installing the Guest Additions is recommended.
This will allow you to display the virtual machine full screen. To do so, click
the ``Devices'' tab at the top of the virtual machine screen, and select the
option ``Install Guest Additions''. The installer will launch automatically and
prompt you. Select ``Run'' and when prompted for a password, enter ``reverse'' and
select ``Authenticate''. A Terminal screen will pop and print the progress of the
installer to the screen. When the installation is complete, hit “Return” to
close the window. For the Guest Additions to take effect, you will need to
restart the virtual machine. To do so, click on the power-button icon on the
upper right corner of the screen and select ``Shut Down'', and select ``Restart''.
Once the machine restarts, log back in and you will now be able to enter full
screen mode (with Ctrl-F or Cmd-F), and increase the window size to your
preference. 
\end{document}
